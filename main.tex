%%%%%%%%%%%%%%%%%%%%%%%%%%%%%%%%%%%%%%%%%%%%%%%%%%%%%%%%%%%%%%%%%%%%%%
% LaTeX Example: Project Report
%
% Source: http://www.howtotex.com
%
% Feel free to distribute this example, but please keep the referral
% to howtotex.com
% Date: March 2011 
% 
%%%%%%%%%%%%%%%%%%%%%%%%%%%%%%%%%%%%%%%%%%%%%%%%%%%%%%%%%%%%%%%%%%%%%%
% How to use writeLaTeX: 
%
% You edit the source code here on the left, and the preview on the
% right shows you the result within a few seconds.
%
% Bookmark this page and share the URL with your co-authors. They can
% edit at the same time!
%
% You can upload figures, bibliographies, custom classes and
% styles using the files menu.
%
% If you're new to LaTeX, the wikibook is a great place to start:
% http://en.wikibooks.org/wiki/LaTeX
%
%%%%%%%%%%%%%%%%%%%%%%%%%%%%%%%%%%%%%%%%%%%%%%%%%%%%%%%%%%%%%%%%%%%%%%
% Edit the title below to update the display in My Documents
%\title{Project Report}
%
%%% Preamble
\documentclass[paper=a4, fontsize=11pt]{scrartcl}
%%%%%%%%%%%%%%%%%%%%%%%%%%%%%%%%%%%%%%%%%%%%%%%%%%%%%%%%%%%%%%%%%%%%%%%
%%% Packages
\usepackage[T1]{fontenc}
\usepackage{fourier}
\usepackage[english]{babel}															% English language/hyphenation
\usepackage[protrusion=true,expansion=true]{microtype}	
\usepackage{amsmath,amsfonts,amsthm} % Math packages
\usepackage[pdftex]{graphicx}	
\usepackage{url}
\usepackage{sectsty}
\allsectionsfont{\centering \normalfont\scshape}
\usepackage{amsmath}
\usepackage{graphicx}
\usepackage{amsthm}
\usepackage{amssymb}
\usepackage[colorlinks=true, allcolors=blue]{hyperref}
\usepackage[dvipsnames]{xcolor}
\newtheorem{theorem}{Theorem}[section]
\newtheorem{corollary}{Corollary}[theorem]
\newtheorem{lemma}[theorem]{Lemma}
\usepackage{float}
\usepackage[dvipsnames]{xcolor}
\usepackage{algorithm}
\usepackage{algpseudocode}
\usepackage[all]{xy}
\usepackage{mathtools}
\usepackage[
backend=biber,
style=alphabetic,
sorting=ynt
]{biblatex}
\usepackage{alltt}
\usepackage{fontawesome}
\usepackage{times}
\usepackage{xspace}
\usepackage{setspace}
\usepackage{enumitem}
\usepackage{amsmath}
\usepackage{amssymb}
\usepackage[dvipsnames]{xcolor}
%%%%%%%%%%%%%%%%%%%%%%%%%%%%%%%%%%%%%%%%%%%%%%%%%%%%%%%%%%%%%%%%%%%%%%%
%%% Custom headers/footers (fancyhdr package)
\usepackage{fancyhdr}
\pagestyle{fancyplain}
\fancyhead{}											% No page header
\fancyfoot[L]{}											% Empty 
\fancyfoot[C]{}											% Empty
\fancyfoot[R]{\thepage}									% Pagenumbering
\renewcommand{\headrulewidth}{0pt}			% Remove header underlines
\renewcommand{\footrulewidth}{0pt}				% Remove footer underlines
\setlength{\headheight}{13.6pt}
%%%%%%%%%%%%%%%%%%%%%%%%%%%%%%%%%%%%%%%%%%%%%%%%%%%%%%%%%%%%%%%%%%%%%%%
%%% Equation and float numbering
\numberwithin{equation}{section}		% Equationnumbering: section.eq#
\numberwithin{figure}{section}			% Figurenumbering: section.fig#
\numberwithin{table}{section}				% Tablenumbering: section.tab#
%%%%%%%%%%%%%%%%%%%%%%%%%%%%%%%%%%%%%%%%%%%%%%%%%%%%%%%%%%%%%%%%%%%%%%%
%%% New Commands
%%% Maketitle metadata
\newcommand{\horrule}[1]{\rule{\linewidth}{#1}} 	% Horizontal rule
\newcommand{\im}{\ensuremath{\operatorname{im}}}
\newcommand{\lm}{\ensuremath{\operatorname{LM}}}
\newcommand{\lt}{\ensuremath{\operatorname{LT}}}
\newcommand{\lc}{\ensuremath{\operatorname{LC}}}
\newcommand{\lev}{\ensuremath{\operatorname{lev}}}
\newcommand{\coker}{\ensuremath{\operatorname{coker}}}
\newcommand{\initTerm}{\ensuremath{\operatorname{in}}}
\newcommand{\mingens}{\ensuremath{\operatorname{mingens}}}
\newcommand{\lex}{\ensuremath{\operatorname{lex}}}
\newcommand{\lcm}{\ensuremath{\operatorname{lcm}}}
\newcommand{\grlex}{\ensuremath{\operatorname{grlex}}}
\newcommand{\grevlex}{\ensuremath{\operatorname{grevlex}}}
\newcommand{\id}{\ensuremath{\operatorname{id}}}
\newcommand{\multideg}{\ensuremath{\operatorname{multideg}}}
\newcommand{\onto}{\ensuremath{\twoheadrightarrow}}
\newcommand{\into}{\ensuremath{\hookrightarrow}}
\newcommand{\Division}{\ensuremath{\operatorname{Division}}}
\newcommand{\Euclidean}{\ensuremath{\operatorname{Euclidean}}}
\newcommand{\tab}[1]{\hspace{.2667\textwidth}\rlap{#1}}
\newcommand{\itab}[1]{\hspace{0em}\rlap{#1}}
\newcommand{\RNum}[1]{\uppercase\expandafter{\romannumeral #1\relax}}

%%% Renew Commands
\renewcommand{\deg}{\ensuremath{\operatorname{deg}}}
\renewcommand{\max}{\ensuremath{\operatorname{max}}}
\renewcommand{\to}{\ensuremath{\rightarrow}}
\renewcommand{\th}{\ensuremath{\operatorname{th}}}

%%% Styles
\theoremstyle{definition}
\newtheorem{definition}{Definition}[section]

\theoremstyle{remark}
\newtheorem*{remark}{\textbf{Remark}}

\theoremstyle{example}
\newtheorem{example}{\textbf{Example}}[section]

\newcommand{\qedwhite}{\hfill \ensuremath{\Box}}

\newtheorem{prop}{Proposition}[section]
%%%%%%%%%%%%%%%%%%%%%%%%%%%%%%%%%%%%%%%%%%%%%%%%%%%%%%%%%%%%%%%%%%%%%%%
\title{
		%\vspace{-1in} 	
		\usefont{OT1}{bch}{b}{n}
		\normalfont \normalsize \textsc{Georgia Tech} \\ [25pt]
		\horrule{0.5pt} \\[0.4cm]
		\huge Comprehensive Exam Review - Algebra \\
		\horrule{2pt} \\[0.5cm]
}
\author{
		\normalfont 								\normalsize
        Ruiqi (Rickey) Huang                 \\\normalsize
        \today
}
\date{}
\addbibresource{references.bib}
%%%%%%%%%%%%%%%%%%%%%%%%%%%%%%%%%%%%%%%%%%%%%%%%%%%%%%%%%%%%%%%%%%%%%%%
%%% Set up
\begin{document}
\maketitle
\thispagestyle{empty}

\newpage

%%%%%%%%%%%%%%%%%%%%%%%%%%%%%%%%%%%%%%%%%%%%%%%%%%%%%%%%%%%%%%%%%%%%%%%
%%% Table of contents
\tableofcontents
\thispagestyle{empty}

\newpage

%%%%%%%%%%%%%%%%%%%%%%%%%%%%%%%%%%%%%%%%%%%%%%%%%%%%%%%%%%%%%%%%%%%%%%%
%%% Begin document
\section{Linear Algebra}

\subsection{Linear Algebra \RNum{1}}

\subsection{Linear Algebra \RNum{2}}

\paragraph{}

I consulted Sheldon Axler's ``Linear Algebra Done Right'' for review purposes \cite{axler_linear_2015}.

\subsubsection{Vector Spaces}

\paragraph{}

$\mathbb{F}$ is a generalization of $\mathbb{R}$ and $\mathbb{C}$, since they are both fields.

\begin{definition}[\textbf{list, length}]
    Suppose that n is a non-negative integer. A \textbf{list} of \textbf{length} $n$ is an ordered collection of $n$ elements. A list of lengths $n$ looks like this:
    \begin{equation}
        (x_1, \cdots, x_n).
    \end{equation}
    Two lists are equal if and only if they have the same length and the same elements in the same order.
\end{definition}

\paragraph{}

lists are always finite. Lists differ from sets in two ways: in lists, order matters and repetitions have meaning; in set, order, and repetitions are irrelevant.

\begin{definition}[$\mathbb{F}^n$]
    $\mathbb{F}^n$ is the set of all the length lists $n$ of elements of $\mathbb{F}$:
    \begin{equation}
        \mathbb{F}^n = \{(x_1,\cdots,x_n):x_j\in \mathbb{F} \text{ for } j = 1, \cdots, n\}.
    \end{equation}
    For $(x_1, \cdots, x_n) \in \mathbb{F}^n$ and $j \in \{1, \cdots, n\}$, we say that $x_j$ is $j^{\th}$ \textbf{coordinate} of $(x_1, \cdots, x_n)$.
\end{definition}

\paragraph{}

The additions in $\mathbb{F}^n$ are defined as element-wise additions, and the commutativity of the addition in $\mathbb{F}^n$ is the same as in $\mathbb{F}$.

\begin{definition}[$0$]
    Let $0$ denote the list of length $n$ whose coordinates are all $0$:
    \begin{equation}
        0 = (0, \cdots, 0)
    \end{equation}
\end{definition}

\paragraph{}

$0$ as a list is an additive identity for $\mathbb{F}^n$
\begin{equation}
    x + 0 = x,\; \forall x \in \mathbb{F}^n
\end{equation}

\paragraph{}

When we think of lists as arrows, we refer to them as \textbf{vectors}.

\begin{remark}
    Vectors are just aids to our understanding, not substitutes for the actual mathematics that we will develop.
\end{remark}

\begin{definition}[\textbf{scalar multiplication in $\mathbb{F}^n$}]
    The product of a number $\lambda$ and a vector in $\mathbb{F}^n$ is computed by multiplying each vector coordinate by $\lambda$:
    \begin{equation}
        \lambda(x_1, \cdots, x_n) = (\lambda x_1, \cdots, \lambda x_n);
    \end{equation}
    Here $\lambda \in \mathbb{F}$ and $(x_1,\cdots,x_n) \in \mathbb{F}^n$.
\end{definition}

In terms of the geometrical meaning of scalar multiples, it represents the transformations that shrink or stretch a vector x by a factor of $\lambda$.

If $\lambda$ is negative and $x$ is a vector in $\mathbb{R}^2$, then $\lambda x$ is the vector that points in the direction opposite to that of $x$ and whose length is $\lvert \lambda \rvert$ times the length of $x$, as shown here.

Before defining the formal definition of the vector spaces, we should first explain the definition of addition and scalar multiplication.

\begin{definition}
    Let $V$ be a set of elements in $\mathbb{F}^n$. An addition in $V$ is a function that assigns an element $u+v\in V$ to each pair of elements $u,v \in V$. A scalar multiplication in a set $V$ is a function that assigns an element $\lambda v \in V$ to each $\lambda \in \mathbb{F}$ and each $v \in V$.
\end{definition}

\begin{definition}
    A vector space is a set $V$ along with an addition on V and a scalar multiplication on V such that the following properties hold:
    \newline
    \textbf{commutativity}
    \begin{equation}
        u + v = v + u \text{ for all } u, v \in V;
    \end{equation}
    \textbf{associativity}
    \begin{equation}
        (u + v) + w = u + (v + w) \text{ and } (ab)v = a(bv) \text{ for all } u, v, w \in V \text{ and all } a, b \in \mathbb{F};
    \end{equation}
    \textbf{additive identity}
    \begin{equation}
        \text{there exists an element } 0 \in V \text{ such that } v + 0 = v, \text{ for all } v \in V;
    \end{equation}
    \textbf{additive inverse}
    \begin{equation}
        \text{for every} v \in V\text{, there exists }w \in V \text{ such that }v + w = 0;
    \end{equation}
    \textbf{multiplicative identity}
    \begin{equation}
        1v = v \text{ for all } v \in V;
    \end{equation}
    \textbf{distributive properties}
    \begin{equation}
        a(u + v) = au + av \text{ and } (a+b)v = av + bv \text{ for all } a,b \in \mathbb{F} \text{ and all } u,v \in V.
    \end{equation}
\end{definition}

\begin{definition}[$\mathbb{F}^{\infty}$]
    $\mathbb{F}^{\infty}$ is defined as the set of all sequences of elements of $\mathbb{F}$:
    \begin{equation}
        \mathbb{F}^{\infty} = \{(x_1,x_2,\cdots)\,:\,x_j\in\mathbb{F} \text{ for } j = 1,2,\cdots\}.
    \end{equation}
    The addition and scalar multiplication in $\mathbb{F}^{\infty}$ are defined as expected
\end{definition}

We introduce a general way to produce a vector space.

\begin{definition}[$\mathbb{F}^{S}$]
    If $S$ is a set,
    \begin{itemize}
        \item Then $\mathbb{F}^{S}$ denotes the set of functions from $S$ to $\mathbb{F}$.
        \item For $f,g \in \mathbb{F}^{S}$, \textbf{sum} $f+g\in \mathbb{F}^{S}$ is the function defined by
        \begin{equation}
            (f+g)(x) = f(x) + g(x)
        \end{equation}
        for all $x \in S$.
        \item For $\lambda \in \mathbb{F}$ and $f \in \mathbb{F}^{S}$, and \textbf{product} $\lambda f \in \mathbb{F}^{S}$ is the function defined by
        \begin{equation}
            (\lambda f)(x) = \lambda f(x)
        \end{equation}
        for all $x \in S$.
    \end{itemize}
\end{definition}

\begin{example}[$\mathbb{F}^{S}$ is a vector space]
    If $S$ is the non-empty set
    \begin{itemize}
        \item The additive identity of $\mathbb{F}^{S}$ is the function $0: S \to \mathbb{F}$ defined by
        \begin{equation}
            0(x) = 0
        \end{equation}
        for all $x \in S$.
        \item For $f \in \mathbb{F}$, the additive inverse of $f$ is the function $-f: S \to \mathbb{F}$ defined by
        \begin{equation}
            (-f)(x) = -f(x)
        \end{equation}
        for all $x \in S$.
    \end{itemize}
\end{example}

\begin{remark}
    $\mathbb{F}^{n}$ and $\mathbb{F}^{\infty}$ are two special cases of $\mathbb{F}^{S}$.
    \begin{equation}
        \begin{aligned}
            \mathbb{F}^{n} &= \mathbb{F}^{1,2,\cdots,n}\\
            \mathbb{F}^{\infty} &= \mathbb{F}^{1,2,\cdots}
        \end{aligned}
    \end{equation}
\end{remark}

\begin{prop}[\textbf{Unique Additive Identity}]
    A vector space has a unique additive identity.
\end{prop}

\begin{proof}
    Suppose by contradiction that $0$ and $0'$ are both additive identities for some vector space $V$. Then
    \begin{equation}
        0 = 0 + 0' = 0' + 0 = 0'.
    \end{equation}
\end{proof}

\begin{prop}[\textbf{Unique Additive Inverse}]
    Every element in a vector space has a unique additive inverse
\end{prop}

\begin{proof}
    Let V be a vector space. Let $v \in V$. Suppose by contradiction that $w$ and $w'$ are two additive inverses of $v$. Then the
    \begin{equation}
        w = w + 0 = w + (v + w') = (w + v) + w' = 0 + w' = w'.
    \end{equation}
\end{proof}

\begin{prop}[\textbf{The Number 0 times a Vector}]
    $0v = 0$ for every $v \in V$
\end{prop}

\begin{proof}
    Let $x$ be the additive inverse of $0v$. Then for all $v \in V$
    \begin{equation}
        \begin{aligned}
            &0v = (0 + 0)v = 0v + 0v\\
            \Rightarrow &0v + x = 0v + 0v + x\\
            \Rightarrow &0 = 0v + (0v + x)\\
            \Rightarrow &0 = 0v + 0\\
            \Rightarrow &0 = 0v
        \end{aligned}
    \end{equation}
\end{proof}

\begin{prop}[\textbf{A Number times Vector 0}]
    $a0 = 0$ for every $a \in \mathbb{F}$
\end{prop}

\begin{proof}
    For all $a \in \mathbb{F}$, let $x$ be the additive inverse of $a0$. We have
    \begin{equation}
        \begin{aligned}
            &a0 = a(0 + 0) = a0 + a0\\
            \Rightarrow &a0 + x = a0 + a0 + x\\
            \Rightarrow &0 = a0 + (a0 + x)\\
            \Rightarrow &0 = a0 + 0\\
            \Rightarrow &0 = a0
        \end{aligned}
    \end{equation}
\end{proof}

\begin{prop}[\textbf{The Number $-1$ times a Vector}]
    $(-1)v = -v$ for every $v \in V$
\end{prop}

\begin{proof}
    For $v \in V$, we have
    \begin{equation}
        \begin{aligned}
            v + (-1)v = 1v + (-1)v = (1+(-1))v = 0v = 0
        \end{aligned}
    \end{equation}
    Hence, $(-1)v$ is the additive inverse of $v$. i.e. $(-1)v = -v$
\end{proof}

\begin{definition}[\textbf{Subspace}]
    A subset $U$ of $V$ is called a \textbf{subspace} of $V$ if $U$ is also a vector space (using the same addition and scalar multiplication as on $V$).
\end{definition}

\begin{prop}
    A subset $U$ of $V$ is a subspace of $V$ if and only if $U$ satisfies the following three conditions:\\
    \textbf{additive identity}
    \begin{equation}
        0 \in U
    \end{equation}
    \textbf{closed under addition}
    \begin{equation}
        u,w \in U \text{ implies } u + w \in U
    \end{equation}
    \textbf{closed under scalar multiplication}
    \begin{equation}
        a \in \mathbb{F} \text{ and } u \in U \text{ implies } au \in U
    \end{equation}
\end{prop}

\begin{proof}
    If $U$ is a subspace of $V$, then $U$ satisfies the three conditions listed above by the definition of the vector space.
    
    Conversely, suppose $U$ satisfies the three conditions above, we want to show that $U$ is a vector space. First the closure conditions make sure the addition and scalar multiplication make sense on $U$. The commutativity, associativity, and distributive properties are automatically fulfilled, since $U \subseteq V$ and $V$ is a vector space. The additive identity is $0$. For an arbitrary element $u \in U$, we could show that $-u$ is the additive inverse of the $u$, since $u + (-u) = (1+(-1))u = 0u = 0$. Hence, $U$ is a vector space.
    \end{proof}

\begin{definition}[\textbf{Sum of Subsets}]
    Suppose $U_1,\cdots,U_m$ are subsets of $V$. The \textbf{sum} of $U_1,\cdots,U_m$, denoted $U_1+\cdots+U_m$, is the set of all possible sums of elements of $U_1,\cdots,U_m$. More precisely,
    \begin{equation}
        U_1+\cdots+U_m = \{u_1+\cdots+u_m:u_1\in U_1,\cdots, u_m\in U_m\}.
    \end{equation}
\end{definition}

\begin{prop}[\textbf{Sum of Subspace is the smallest containing Subspace}]
    Suppose $U_1,\cdots,U_m$ are subspaces of $V$. Then $U_1+\cdots+U_m$ is the smallest subspace of $V$ containing $U_1,\cdots,U_m$.
\end{prop}

\begin{proof}
    It is clear that $U_1+\cdots+U_m$ is a subspace of $V$, since $0 \in U_1+\cdots+U_m$, since $0 = 0 + \cdots + 0$ for each $0$ from each $U_1, \cdots, U_m$, and $U_1+\cdots+U_m$ is closed under addition and scalar multiplication because $U_i \in V$ for all $i = 1,\cdots,m$ and $V$ is closed under addition.
    
    Clearly, each $U_i$ for $i = 1,\cdots,m$ is contained in $U_1+\cdots+U_m$ by setting corresponding elemens to $0$. Next, it suffices to show that every subspace of $V$ containing $U_1,\cdots,U_m$ contains $U_1+\cdots+U_m$. Indeed, every subspace containing $U_1,\cdots,U_m$ must contain every finite sum of elements in $U_1,\cdots,U_m$.
\end{proof}

\paragraph{}

Sums of subspaces in the theory of vector spaces are analogous to unions of subsets in set theory. Given two subspaces of a vector space, the smallest subset containing them is their sum. Analogously, given two subset of a set, the smallest subset containing them is their union. similarly, we are going to introduce an analogy of disjoint union in the theory of vector spaces.

\begin{definition}[\textbf{Direct Sums}]
    Suppose $U_1,\cdots,U_m$ are subspaces of $V$.
    \begin{itemize}
        \item The sum $U_1+\cdots U_m$ is called \textbf{direct sum} if each element of $U_1+\cdots+U_m$ can be uniquely written as a sum $u_1+\cdots+u_m$, where each $u_j$ is in $U_j$.
        \item If $U_1+\cdots+U_m$ is a direct sum, then $U_1\bigoplus\cdots\bigoplus U_m$ denotes $U_1+\cdots+U_m$, with the $\bigoplus$ notation serving as an indication that this is a direct sum.
    \end{itemize}
\end{definition}

\newpage
\section{Abstract Algebra}

I consulted Dummit and Foote's ``Abstract Algebra'' \cite{foote_abstract_2003} and Aluffi's ``Algebra Chapter 0''\cite{aluffi_algebra_2009} for reviewing purpose .

\subsection{Preliminaries}

\subsubsection{Basics}

\begin{definition}[\textbf{Injection}]
    Let $f: A \to B$, $f$ is injective or is an injection if whenever $a_1 \neq a_2$, then $f(a_1)\neq f(a_2)$.
\end{definition}

\begin{definition}[\textbf{Surjection}]
    Let $f: A \to B$, $f$ is surjective or is a surjection if for all $b \in B$ there is some $a \in A$ such that $f(a) = b$, i.e., the image of $f$ is all of $B$. Notice that since a function always maps onto its range, it is necessary to specify the codomain $B$ in order for the question of surjectivity to be meaningful.
\end{definition}

\begin{definition}[\textbf{Bijection}]
    Let $f: A \to B$, $f$ is bijective or is a bijection if it is both injective and surjective. If such a bijection exist between $A$ and $B$, we say that $A$ and $B$ are in bijective correspondence.
\end{definition}

\begin{definition}[\textbf{Left Inverse}]
    Let $f: A \to B$. $f$ has a left inverse if there is a function $g: B \to A$ such that $g \circ f: A \to A$ is the identity map, i.e., $(g \circ f)(a) = a$ for all $a \in A$. 
\end{definition}

\begin{definition}[\textbf{Right Inverse}]
    Let $f: A \to B$. $f$ has a left inverse if there is a function $g: B \to A$ such that $f \circ g: A \to A$ is the identity map, i.e., $(f \circ g)(a) = a$ for all $a \in A$. 
\end{definition}

\begin{prop}
    Let $f: A \to B$.
    \begin{itemize}
        \item The map $f$ is injective if and only if $f$ has a left inverse
        \item The map $f$ is surjective if and only if $f$ has a right inverse
        \item The map $f$ is bijective if and only if there is a function $g: B \to A$ such that $f \circ g$ and $g \circ f$ are both identity maps.
        \item If $A$ and $B$ are finite sets with the same number of elements, then $f$ is bijective if and only if $f$ is surjective if and only if $f$ is injective.
    \end{itemize}
\end{prop}

\begin{proof}
    Suppose $f$ is injective, and let $C = \im f$. Then, define a function 
    \begin{equation}
        \begin{aligned}
            h: &C \to A\\
            &c \to f^{-1}(c)
        \end{aligned}
    \end{equation}
    $h$ is well-defined, since $f$ is injective, we know that for all $c \in C$, $f^{-1}(c)$ is unique. Then for all $g: B \to A$ such that $g$ is an extension of $h$ to $B$ could be the left inverse of $g$. Indeed, for all $a \in A$, $(g \circ f)(a) = g(f(a)) = f^{-1}(f(a)) = a$, since $f(a) \in C$. 
    
    Next, suppose $f$ has a left inverse $g: B \to A$. Then $(g \circ f)(a) = a$ for all $a \in A$. Suppose by contradiction that $f$ is not injective, then there is $a_1, a_2 \in A$ such that $a_1 \neq a_2$ but $f(a_1) = f(a_2)$. Then $a_1 = (g \circ f)(a_1) = g(f(a_1)) = g(f(a_2)) = (g \circ f)(a_2) = a_2$. This lead to a contradiction to $a_1 \neq a_2$. Hence, $f$ is injective.
    
    The similar proofs for the rest of the bullet points.
\end{proof}

\begin{remark}
    In the situation of the third bullet point in the previous proposition, the function $g$ is necessarily to be unique, and we shall call it to be the two-sided inverse (or simply \textbf{inverse}) of the function $f$.
\end{remark}

\begin{definition}[\textbf{Permutation}]
    A permutation of a set $A$ is simply a bijection from $A$ to $A$.
\end{definition}

\begin{definition}[\textbf{Restriction}]
    If $A \subseteq B$ and $f: B \to C$, we denote the \textbf{restriction} of $f$ to $A$ by $f\lvert A$. When the domain we are considering is understood, we shall occasionally denote $f \lvert A$ again simply as $f$ even though they are formally different functions (their domains are different).
\end{definition}

\begin{definition}[\textbf{Extension}]
    If $A \subseteq B$ and $g: a \to C$, and there exist a function $f: B \to C$ such that $f\lvert A = g$, we shall say $f$ is an extension of $g$ to $B$ (such a map need not to exist nor be unique).
\end{definition}

\begin{definition}[\textbf{Binary Relation}]
    A \textbf{binary relation} on a set $A$ is a subset $R$ of $A \times A$ and we write $a \sim b$ if $(a,b) \in R$.
\end{definition}

\begin{definition}[\textbf{Equivalence Relation}]
    A relation $\sim$ on $A$ is said to be \textbf{equivalence relation} if it is reflexive, symmetric, and transitive.
\end{definition}

\begin{definition}[\textbf{Equivalence Class}]
    If $\sim$ defines an equivalence relation on $A$, then the equivalence class of $a \in A$ is defined to be $\{x \in A \lvert x \sim a\}$. Elements of the equivalence class of $a$ are said to be equivalent to $a$. If $C$ is said to be an equivalence class, then any elements of $C$ are \textbf{representative} of $C$.
\end{definition}

\begin{definition}[\textbf{Partition}]
    A partition of $A$ is any collection $\{A_i \lvert i \in I\}$ of nonempty subsets of $A$ ($I$ some indexing set) such that
    \begin{itemize}
        \item $A = \bigcup_{i\in I}A_i$, and
        \item $A_i \cap A_j = \emptyset$, for all $i,j \in I$ with $i \neq j$, i.e., $A$ is the disjoint union of the sets in the partition.
    \end{itemize}
\end{definition}

\begin{prop}
    Let $A$ be a non-empty set.
    \begin{itemize}
        \item If $\sim$ defines an equivalence relation on $A$, then the set of equivalence classes of $\sim$ forms a partition of $A$.
        \item If $\{A_i \lvert i \in I\}$ is a partition of $A$, then there is an equivalence relation on $A$ whose equivalence classes are precisely the sets $A_i, \, i \in I$.
    \end{itemize}
\end{prop}

\begin{remark}
    The proposition above claims that the equivalence classes and the partitions define the same thing.
\end{remark}

\newpage
\subsubsection{Properties of the integers}

\begin{prop}
    Below are some basic properties of the integers $\mathbb{Z}$
    \begin{itemize}
        \item (Well Ordering of $\mathbb{Z}$) If $A$ is some nonempty subset of $\mathbb{Z}^{+}$, there is some element $m \in A$ such that $m \leq A$, for all $a \in A$ ($m$ is called a \textbf{minimal element} of $A$).
        \item If $a,b \in \mathbb{Z}$ with $a \neq 0$, we say $a$ \textbf{divides} $b$ if there is an element $c \in \mathbb{Z}$ such that $b = ac$. In this case we write $a \lvert b$; if $a$ does not divide $b$, we write $a \nmid b$.
        \item If $a,b \in \mathbb{Z}-\{0\}$, there is a unique positive integer $d$, called the \textbf{greatest common divisor} of $a$ and $b$ (or $\gcd$ of $a$ and $b$)
        \item If $a,b \in \mathbb{Z}-\{0\}$, there is a unique positive integer $d$, called the \textbf{least common multiple} of $a$ and $b$ (or $\lcm$ of $a$ and $b$)
        \item $ab = \gcd(a,b)\lcm(a,b)$
    \end{itemize}
\end{prop}

\begin{theorem}[\textbf{Division Algorithm}]
    If $a,b \in \mathbb{Z}-\{0\}$, then there exist unique $q,r \in \mathbb{Z}$ such that
    \begin{equation}
        a = qb +r \text{ and } 0 \leq r < \lvert b \rvert,
    \end{equation}
    where $q$ is the \textbf{quotient} and $r$ the \textbf{remainder}.
\end{theorem}

\begin{proof}
    \textit{Existence}: We could give an algorithm as shown in the Algorithm \ref{alg:divAlg} to prove that there always exist such $q$ and $r$
    \begin{algorithm}[H]
        \caption{Division$[a,b,q_0]$}\label{alg:divAlg}
        \begin{algorithmic}
            \State Input: two integers $a$ and $b$, with $a \geq b$ and an integer $q_0$ satisfying $bq_0 \leq a$
            \State Output: the quotient $q$ and the remainder $r$
            \newline
            \State $q := q_0$
            \State $r := a - bq$
            \While{$r \geq \lvert b \rvert$}
                \State $q = q+1$
                \State $r = r-b$
            \EndWhile
            \State return $q,\, r$
        \end{algorithmic}
    \end{algorithm}
    Because $q$ and $r$ was computed in pairs by the algorithm and $r$ is descending by each iteration of the algorithm, we know the while loop indeed terminates. Also, since the condition of the while loop state that $r \leq \lvert b \rvert$, we know that when the while loop terminates, $r < \lvert b \rvert$.
    \newline
    \textit{Uniqueness}: Suppose by contradiction that there exist $p$ and $s$ with $p \neq q$ and $s \neq r$ such that $a = pb + s$ and $0 \leq s < \lvert b \rvert$. Then $qb + r = pb + s$, which leads $r - s = (p-q)b$. This implies $\lvert r - s \rvert = \lvert (p - q)b \rvert \leq \lvert b \rvert$. However, since $0 \leq r,s < b$, $\lvert r - s \rvert \leq b$ would lead to contradiction. Hence, the $q,r$ given by the Algorithm \ref{alg:divAlg} is unique.
\end{proof}

\begin{theorem}[Euclidean Algorithm]
    The \textbf{Euclidean Algorithm} \ref{alg:eucAlg} is an important procedure which produces the $\gcd$ of two integers $a$ and $b$ by iterating \textbf{Division Algorithm}
    \begin{algorithm}[H]
        \caption{Euclidean$[a,b]$}\label{alg:eucAlg}
        \begin{algorithmic}
            \State Input: two integers $a$ and $b$ with $a \geq b$
            \State output: the greatest common divisor of $a$ and $b$, $g = \gcd(a,b)$
            \newline
            \State $(q,r0) := \Division[a,b,q_0]$ where $q_0$ is an integer satisfying $bq_0 \leq a$
            \State $r := r0$
            \While{$r \neq 0$}
                \State $r = r0$
                \State $(q,r0) = \Division[b,r,h]$ where $h$ is an integer satisfying $bh \leq a$, and $h$ should be updated for each iteration
            \EndWhile
            \State return $r0$
        \end{algorithmic}
    \end{algorithm}
\end{theorem}

\begin{proof}
    The Algorithm \ref{alg:eucAlg} terminates since $r_0$ at each time of iteration creates a decreasing sequence of positive integers. Because the sequence could not continue indefinitely, $r$ would finally be $0$ at a certain time of iteration. Hence $r_0$ exits. It is clear that the output of the algorithm divides both $a$ and $b$ by collecting common terms. Then, suppose by contradiction that there is a common divisor of $a$ and $b$, $s$ that is larger than the output of the algorithm $r$. Then, $s$ divides each $r_0$ produced by the iterations of the algorithm. In particular, $s \mid r$, which implies $s \leq r$. This leads to a contradiction to $s > r$. Hence, $r$ is $\gcd(a,b)$.
\end{proof}

\begin{theorem}
    $\gcd(a,b)$ is a $\mathbb{Z}$-linear combination of $a$ and $b$.
\end{theorem}

\begin{proof}
    Recursively writing the element $r_n$ in Algorithm \ref{alg:eucAlg} in terms of the previous remainders (namely, use equation ($n$) to solve for $r_n = r_{n-2} - q_nr_{n-1}$ in terms of the remainders $r_{n-1}$ and $r_{n-2}$, then use equation $(n-1)$ to write $r_n$ in terms of the remainders $r_{n-2}$ and $r_{n-3}$, etc., eventually writing $r_n$ in terms of $a$ and $b$) 
\end{proof}

\begin{example}
    Suppose $a = 57970$ and $b = 10353$. We could find the greatest common divisor via Algorithm \ref{alg:eucAlg}.
        \begin{align}
            57970 &= (5)10353 + 6205 \label{eqn:2.3}\\
            10353 &= (1)6205 + 4148 \label{eqn:2.4}\\
            6205 &= (1)4148 + 2057 \label{eqn:2.5}\\
            4148 &= (2)2057 + 34 \label{eqn:2.6}\\
            2057 &= (60)34 + 17 \label{eqn:2.7}\\
            34 &= (2)17
        \end{align}
    Hence, $\gcd(a,b) = 17$
    By Equation \ref{eqn:2.7}, we could write $17 = 2057 + (-60)34$. By Equation \ref{eqn:2.6}, we could write $34 = 4148 + (-2)2057$, and substituting this into the previous expression we have $17 = (-60)4148 + (121)2057$. Similarly, we get $2057 = 6205 + (-1)4148$ from Equation \ref{eqn:2.5}, and therefore, we have $17 = (121)6205 + (-181)4148$. Then, we have $17 = (-181)10353 + (302)6205$, and finally, we have $17 = (302)57970 + (-1691)10353 = 302a + (-1691)b$
\end{example}

\begin{remark}
    The $\mathbb{Z}$-combination of $a$ and $b$ for $\gcd(a,b)$ is not unique.
\end{remark}

\begin{definition}[\textbf{Prime and Composite}]
    An element $p$ of $\mathbb{Z}^{+}$ is called a \textbf{prime} if $p > 1$ and the only positive divisor of $p$ are $1$ and $p$ (initially, the word prime will refer only to positive integers). and integer $n > 1$ is said to be \textbf{composite} if it is not prime.
\end{definition}

\begin{prop}
    If $p$ is a prime number and $p\lvert ab$ for some $a,b\in\mathbb{Z}$, then either $p \lvert a$ or $p \lvert b$.
\end{prop}

\begin{proof}
    Suppose by contradiction that $p \nmid a$ and $p \nmid b$. Then, $a = np + m$ and $b = sp + t$ for some $n,m,s,t \in \mathbb{Z}$ and $m,t < p$. Hence $ab = (np+m)(sp+t) = nsp + (nt+ms)p + mt$. We know $p \nmid mt$, since otherwise some divisor of $m$ or $t$ would be a divisor of $p$ other than $1$ and $p$. This contradicts the hypothesis that $p$ is prime. Hence, $p \nmid ab$. This is a contradiction. Therefore, either $p \lvert a$ or $p \lvert b$.
\end{proof}

\begin{theorem}[\textbf{Fundamental Theorem of Arithmetic}]
    If $n \in \mathbb{Z}, \; n > 1$, then $n$ can be factored uniquely into the product of primes, i.e., there are distinct primes $p_1,p_2,\cdots,p_s$ and positive integers $\alpha_1,\alpha_2,\cdots,\alpha_s$ such that
    \begin{equation}
        n = p_1^{\alpha_1}p_2^{\alpha_2}\cdots p_s^{\alpha_s}.
    \end{equation}
    This factorization is unique by following a certain order to arrange the factors $p_1,p_2,\cdots,p_s$.
\end{theorem}

\begin{prop}
    Suppsoe the positive integers $a$ and $b$ are expressed as products of prime powers:
    \begin{equation}
        a = p_1^{\alpha_1}p_2^{\alpha_2}\cdots p_s^{\alpha_s}, \;\; b = p_1^{\beta_1}p_2^{\beta_2}\cdots p_s^{\beta_s}
    \end{equation}
    where $p_1, p_2,\cdots, p_s$ are distinct and the exponents are $\geq 0$. Then, the greatest common divisor of $a$ and $b$ is
    \begin{equation}
        \gcd(a,b) = p_1^{\min(\alpha_1,\beta_1)}p_2^{\min(\alpha_2,\beta_2)}\cdots p_s^{\min(\alpha_s,\beta_s)},
    \end{equation}
    and
    \begin{equation}
        \gcd(a,b) = p_1^{\max(\alpha_1,\beta_1)}p_2^{\max(\alpha_2,\beta_2)}\cdots p_s^{\max(\alpha_s,\beta_s)}
    \end{equation}
\end{prop}

\begin{definition}[\textbf{Euler $\phi$-function}]
    The \textbf{Euler $\phi$-function} is defined as follows: for $n \in \mathbb{Z}^{+}$ let $\phi(n)$ be the number of positive integer $a \leq n$ with $a$ relatively prime to $n$, i.e., $\gcd(a,n) = 1$.
\end{definition}

\begin{prop}
    For primes $p$, $\phi(p) = p - 1$
\end{prop}

\begin{proof}
    Since for all primes $p$ and $n \in \mathbb{Z}^{+}, n \leq p$, only when $n = p$, $\gcd(n,p) \neq 1$. Hence, $\phi(p) = p-1$.
\end{proof}

\begin{prop}
    For primes $p$, and for all $a \geq 1$ we have the formula
    \begin{equation}
        \phi(p^a) = p^a - p^{a-1} = p^{a-1}(p-1)
    \end{equation}
\end{prop}

\begin{proof}
    Since for any $p^a$ the integers $n \in \mathbb{Z}^{+}, n \leq p$ such that $\gcd(n,p) = 1$ are those $p \nmid n$. Hence, there are $p-1$ kind of remainder when dividing such $n$ by $p$. For any single remainder, there are $\tfrac{p^a}{p} = p^{a-1}$ many integers less than $p$. Hence, there are $p^{a-1}(p-1)$ many $n$. Therefore, $\phi(p^a) = p^{a-1}(p-1)$.
\end{proof}

\begin{prop}
    The function $\phi$ is multiplicative in the sense that
    \begin{equation}
        \phi(ab) = \phi(a)\phi(b) \quad \text{if } \gcd(a,b) = 1,
    \end{equation}
    note that it is important here that $a$ and $b$ be relatively prime.
\end{prop}

\begin{lemma}
    The general formula for the values of $\phi$: if $n = p_1^{\alpha_1}p_2^{\alpha_2}\cdots p_s^{\alpha_s}$, then 
    \begin{equation}
        \begin{aligned}
            \phi(n) &= \phi(p_1^{\alpha_1})\phi(p_2^{\alpha_2})\cdots \phi(p_s^{\alpha_s})\\
            &= p_1^{\alpha_1-1}(p_1-1)p_2^{\alpha_2-1}(p_2-1)\cdots p_s^{\alpha_s-1}(p_s-1)
        \end{aligned}
    \end{equation}
\end{lemma}

\newpage
\subsection{Group Theory}



%%% End document

\newpage
\printbibliography

\end{document}